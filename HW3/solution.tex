\documentclass[11pt]{article}

\usepackage{fancyhdr}
\usepackage{graphicx}
\usepackage{geometry}
\usepackage{lastpage}
\usepackage{titling}
\usepackage{sectsty}
\usepackage{setspace}
\usepackage{changepage}
\usepackage[shortlabels]{enumitem}
\usepackage{subcaption}
\usepackage{helvet}
\usepackage{hyperref}

\usepackage{tabularx}
\usepackage[table]{xcolor}
\usepackage{array}
\newcolumntype{P}[1]{>{\centering\arraybackslash}p{#1}}

\usepackage{siunitx}
\usepackage{nicefrac}
\usepackage{amsmath}
\usepackage{gensymb}
\usepackage{amssymb}
\usepackage{float}
\setcounter{MaxMatrixCols}{11}
\usepackage{indentfirst}

\usepackage{listings}
\usepackage{matlab-prettifier}
% \usepackage{color}
% \definecolor{dkgreen}{rgb}{0,0.6,0}
% \definecolor{gray}{rgb}{0.5,0.5,0.5}
% \definecolor{mauve}{rgb}{0.58,0,0.82}

\lstset
{
  frame=tb,
  style=Matlab-editor,
  % language=MATLAB, %Matlab-editor,
  aboveskip=3mm,
  belowskip=3mm,
  showstringspaces=false,
  columns=flexible,
  basicstyle={\small\ttfamily},
  numbers=none,
  % numberstyle=\tiny\color{gray},
  % keywordstyle=\color{blue},
  % commentstyle=\color{dkgreen},
  % stringstyle=\color{mauve},
  breaklines=true,
  breakatwhitespace=true,
  tabsize=3
}

\geometry
{
  letterpaper, 
  total={175.9mm,229.4mm}, 
  top=25mm, 
  left=20mm, 
  headheight=15pt,
  voffset=12pt,
  footskip=15pt
}
\author{Daniel Sturdivant}
\title{Homework 3}
\date{April 2023}
\graphicspath{ {./media/} }

\pagestyle{fancy}
\fancyhead[R]{April 10, 2023}
\fancyhead[L]{Sturdivant, Daniel}
\fancyhead[C]{MECH 7710 Optimal}
\fancyfoot[C]{Page \thepage\ of \pageref{LastPage}}

\makeatletter
\def\@maketitle
{
  \null
  \begin{center}
    {\huge \@title \\}
  \end{center}
  \vskip 5mm
}
\makeatother

\sectionfont{\fontsize{16}{16}}
\subsectionfont{\fontsize{13}{13}\normalfont}
\renewcommand{\thesubsection}{\arabic{section}-\arabic{subsection}}
\renewcommand{\familydefault}{\sfdefault}
\newcommand{\solution}{\textbf{Solution: \\}}


%% ========================================================================== %%
\begin{document}

\maketitle
\thispagestyle{fancy}
\setstretch{1.25}
% \setlength{\parskip}{0em}
% \setlength{\abovedisplayskip}{-8pt}
% \setlength{\belowdisplayskip}{12pt}
\setlength{\parindent}{0pt}

\begin{enumerate}[label=\textbf{\arabic*.}]
  \itemsep 24pt
  % PROBLEM 1
  \item Kalman Filter at its best – simulation (actually the Kalman filter is 
  also quite reliable when we have an excellent model and low noise sensors). 
  Suppose we have a $2^{nd}$ order system that we are regulating about zero 
  (position and velocity) by wrapping an "optimal" control loop around the 
  system. The new dynamics of the continuous time system are given by the 
  closed-loop $A$ matrix:
  \begin{equation*}
    A_{cl} = 
    \begin{bmatrix}
      0 & 1 \\ -1 & -1.4
    \end{bmatrix}
  \end{equation*}
  Suppose our measurement is simply position ($C=\begin{bmatrix} 1 & 0 
  \end{bmatrix}$). There is a white noise process disturbance (force, $B_w = 
  \begin{bmatrix} 0 & 1 \end{bmatrix}^T$) acting on the controlled system.
  \begin{enumerate}[(a)]
    \itemsep -2pt 
    \item Simulate the controlled system with the disturbance force 
    ($1\sigma=2$) and a sampled sensor noise ($1\sigma=1$) for 100 seconds at a 
    10 Hz sample rate.
    \item What is $Q$, $Q_d$ and $R_d$?
    \item Calculate the steady state Kalman gain for the system. This can be 
    done in one of many ways: iterate the kalman filter until it converges, 
    dlqe.m, dare.m, kalman.m, dlqr.m (+ predictor to current estimator trick), 
    etc. What is the steady state covariance of the estimates after the time 
    update, $P^{(-)}$, as well as after the measurement update, $P^{(+)}$. 
    Where are the poles of the estimator?
    \item Now use the steady state kalman filter to generate an estimate 
    ($\hat{x}$ and $\hat{\dot{x}}$) of the 2 states over time. Calculate the 
    norm of the standard deviation of the errors for each state.
    \begin{equation*}
      N = \sqrt{( std(\dot{x} - \hat{\dot{x}}) )^2 + ( std(x - \hat{x}) )^2}
    \end{equation*}
    \item Change the ratio of the $Q_d$ and $R_d$ weights in the Kalman filter 
    design (repeat \emph{part d} with the new Kalman gain but \textbf{DO NOT} 
    regenerate a new $x$ and $\dot{x}$) and determine the effect on the 
    estimation errors. For what ratio of $Q_d$ to $R_d$ are the errors 
    minimized? Note: Often in practice we do not know the actual $Q_d$ and 
    $R_d$, so these tend to be "tuning" parameters we can use to tune our 
    filter. However, according to Kalman the estimation errors are only 
    minimized if we use the $Q_d$ and $R_d$ of the physical system.
  \end{enumerate}
  \solution

  % PROBLEM 2
  \item Download the data \emph{hw3\_2} from the website. The data is in the 
  form $\begin{bmatrix} t & y \end{bmatrix}$. Suppose we want to design an 
  estimator to estimate the bias in the measurement $y$. We believe that the 
  bias ($x$) is constant, so we use the model given by:
  \begin{equation*}
    \begin{split}
      \dot{x} &= 0 \\
      y_k &= x_k + \nu_k \\
      \nu_k &\sim N(0,1)
    \end{split}
  \end{equation*}
  \begin{enumerate}[(a)]
    \itemsep -2pt 
    \item Run the Kalman filter estimator with $Q_d = 0$. What happens at 
    $t > 100$ seconds? Why? Calculate the steady state Kalman gain $L_ss$. 
    Plot $L(k)$. This is known as the filter "going to sleep" (becomes a least 
    squares estimator).
    \item To offset this problem we will "tune" $Q_d$ to track the bias. What is 
    the effect of changing $Q_d$ on the ability to track the step change in the 
    bias? Try values of $Q_d$ from $0.0001$ to $0.01$ and plot $L(k)$ as well 
    as the estimate of the bias ($\hat{x}$). What is the tradeoff?
    \item Now filter the measurement using the first order low-pass filter: \\
    (Command: $yf = filter(numd, dend, y, y0)$)
    \begin{equation*}
      H(z) = \dfrac{\sqrt{Q_d}}{z - (1-\sqrt{Q_d})}
    \end{equation*}
    \item How does this compare to the Kalman filter solution. Why are these two 
    filters the same for this problem?
  \end{enumerate}
  \solution

  % PROBLEM 3
  \item Design a "Navigation" type Kalman filter to estimate the states [East, 
  North, Radar\_Bias, Psi, Gyro\_Bias]. Note: this is a non-linear problem that 
  requires an Extended Kalman Filter (EKF) to do correctly. However, we can solve 
  the problem in one of two ways: 
  \begin{enumerate}[(i)]
    \itemsep -2pt
    \item linearize the equations about the nominal operating point and produce 
    a constant A matrix for that operating point
    \item simply update the A matrix at every time step with our measurements or 
    estimates
  \end{enumerate}
  Download the data \emph{hw3\_3} from the website and run the filter sampled at 
  5 Hz.
  \begin{enumerate}[(a)]
    \itemsep -2pt
    \item How did you choose the covariance values for $Q_d$ (especially for the 
    radar and gyro biases).
    \item How does the bias estimation compare to a Least Squares Solution. How 
    does the bias estimate compare to the Recursive Least Squares solution if 
    you make the covariance ($Q_d$) of the bias estimates equal to zero.
    \item Integrate the last 40 seconds of data to see how well you have 
    estimated the biases. This can simply be done by “turning off” the 
    measurements in the observation matrix! Why do the bias estimates remain 
    constant during the 40 second "outage?"
  \end{enumerate}
  \solution

  % PROBLEM 4
  \item Estimator for Vehicle Dynamics. The yaw dynamics of a car (for a 
  stability control system) can be described by the following model (at 
  $25 m/s$):
  \begin{equation*}
    \begin{split}
      \dot{x} &= 
      \begin{bmatrix}
        -2.62 & 12 \\ -0.96 & -2
      \end{bmatrix}
      \begin{bmatrix}
        \dot{\psi} \\ \beta 
      \end{bmatrix}
      + 
      \begin{bmatrix}
        14 \\ 1
      \end{bmatrix} 
      \delta \\
      y_k &= 
      \begin{bmatrix}
        1 & 0
      \end{bmatrix}
      x_k + \nu_k
    \end{split}
  \end{equation*}
  Where:
  \begin{equation*}
    \begin{split}
      \dot{\psi} &= \text{Vehicle Yaw Rate} \\
      \beta &= \text{Vehicle Sideslip Angle} \\
      \delta &= \text{Steer Angle} \\
      \nu_k &= \text{Sample Sensor Noise}
    \end{split}
  \end{equation*}
  \begin{enumerate}
    \itemsep -2pt 
    \item Assuming we can only measure the yaw rate ($\nu_k \sim N 
    [0,(0.1)^2]$), design a Kalman filter to do full state estimation (select a 
    reasonable $Q_d$). Provide a unit step steer input and estimate both states. 
    On one page plot the actual states and estimated states (use 
    $subplot(2,2,n)$ for each of the two states). Where are the steady state
    poles of the estimator?
    \item Now, somebody has loaded the trunk of the vehicle with bricks, 
    changing the CG of the vehicle so now the actual model (at 25 m/s) is:
    \begin{equation*}
      \begin{split}
        \dot{x} &= 
        \begin{bmatrix}
          -2.42 & 4 \\ -0.99 & -2
        \end{bmatrix}
        \begin{bmatrix}
          \dot{\psi} \\ \beta
        \end{bmatrix} 
        +
        \begin{bmatrix}
          15 \\ 1
        \end{bmatrix}
        \delta \\
        y_k &= 
        \begin{bmatrix}
          1 & 0
        \end{bmatrix}
        x_k + \nu_k
      \end{split}
    \end{equation*}
    NOTE: We do not know that somebody has loaded the trunk and that the C.G.
    has changed, therefore we must use the model for \emph{part a} in our Kalman 
    Filter. Redo \emph{part a}. Can you estimate the slip angle correctly? Try 
    various $Q_d$.
    \item Now lets say we have a noisy measurement of the slip angle ($\eta_k 
    \sim N[0,(0.5)^2]$):
    \begin{equation*}
      y_k = 
      \begin{bmatrix}
        1 & 0 \\ 0 & 1
      \end{bmatrix}
      x_k + 
      \begin{bmatrix}
        \nu_k \\ \eta_k
      \end{bmatrix}
    \end{equation*}
    Assuming the sensor noises are uncorrelated, what is $R$?
    \item Redo \emph{part a}. What is the effect of changing the element of 
    $Q_d$ associated with the slip angle estimate. What must $Q_d$ equal to 
    ensure an unbiased estimate of the states. How much filtering does that 
    provide?
  \end{enumerate}
  \solution 

\end{enumerate}

\end{document}